\documentclass[14pt, a4paper]{extarticle}

\usepackage[utf8]{inputenc}
\usepackage[T1]{fontenc}
\usepackage{amsmath}
\usepackage{amssymb}
\usepackage{graphicx}
\usepackage[left=2.00cm, right=2.00cm, top=2.00cm, bottom=2.00cm]{geometry}
\usepackage[russian]{babel}

\usepackage{setspace}
\usepackage{fancyhdr}

\graphicspath{{img/}}

\RequirePackage{caption}
\captionsetup[figure]{justification=centering,name=Рисунок,labelsep=endash}

\usepackage{indentfirst}

\usepackage{float}

\begin{document}
	\onehalfspacing
	\begin{titlepage}
	\begin{center}
		\begin{small}
			\textbf{Министерство науки и высшего образования Российской Федерации}

			ФЕДЕРАЛЬНОЕ ГОСУДАРСТВЕННОЕ АВТОНОМНОЕ ОБРАЗОВАТЕЛЬНОЕ УЧРЕЖДЕНИЕ ВЫСШЕГО ОБРАЗОВАНИЯ
			
			\textbf{<<НАЦИОНАЛЬНЫЙ ИССЛЕДОВАТЕЛЬСКИЙ УНИВЕРСИТЕТ ИТМО>>}
		\end{small}
		
		\vspace{8em}
		
		Отчет по лабораторной работе №4
		
		СИНТЕЗ ОПТИМАЛЬНОГО УПРАВЛЕНИЯ. ПРИНЦИП МАКСИМУМА
		
		По дисциплине <<Оптимальное управление>>
	\end{center}
	
	\vspace{8em}
	
	\begin{flushright}
		Выполнил:\\
		студент группы R42331c\\
		Манахов~С.П.
		
		\vspace{1em}
		
		Преподаватель:\\
		Парамонов~А.В.
	\end{flushright}

	\vfill
	
	\begin{center}
		\small
		Санкт-Петербург\\
		2022 г.\\
	\end{center}
\end{titlepage}
	\setcounter{page}{2}
	
	\section*{Задание}
	
	Найти минимум критерия качества для статической задачи оптимизации:
	\begin{enumerate}
		\item Поиск глобального минимума $J(x,u)$ на основе необходимого и достаточного условия:
		\begin{enumerate}
			\item[1.1] Без ограничений;
			\item[1.2] С ограничением в виде равенства $c(x,u)=0$;
			\item[1.3] С ограничением в виде неравенства $c(x,u)\le0$; 
		\end{enumerate}
		\item Градиентный поиск минимума критерия качества:
		$$J_1(x,u)=J(x,u)$$
		\begin{enumerate}
			\item[2.1] Методом Ньютона Рафсона произвести пошаговый расчет экстремума;
			\item[2.2] Методом наискорейшего спуска для двух различных $\gamma$ (соответствующей колебательной и апериодической сходимости) произвести пошаговый расчет экстремума. 
		\end{enumerate}
	\end{enumerate}
	\begin{table}[H]
		\centering
		\begin{tabular}{|c|c|c|}
			\hline
			Вариант & $J(x,u)$ & $c(x,u)$ \\\hline
			9 & $J(x,u)=4x^2+5u^2+7xu+5x+6u-8$ & $x-u^2-8$ \\\hline
		\end{tabular}
	\end{table}
	
	\newpage
	
	\section*{Описание работы}
	
	Для поиска глобального минимума $J(x,u)$ без ограничений составим градиент:
	$$gradJ(x,u)=\nabla J(x,u)=\left[\begin{matrix}
		\frac{\partial J(x,u)}{\partial x} \\
		\frac{\partial J(x,u)}{\partial u} \\
	\end{matrix}\right]=\left[\begin{matrix}
		8x+7u+5 \\
		10u+7x+6 \\
	\end{matrix}\right]$$
	Приравняем каждый элемент градиента к нулю и решим следующую систему уравнений:
	$$\begin{cases}
		8x+7u+5=0 \\
		10u+7x+6=0 \\
	\end{cases}$$
	Откуда следует, что $x=-\frac{8}{31}$ и $u=-\frac{13}{31}$, подставим в выражение для критерия качества и получим:
	$$J_{min}=-\frac{307}{31}$$
	Для поиска глобального минимума $J(x,u)$ с ограничением в виде равенства $c(x,u)=0$ составим лагранжиан:
	$$L(x,u,\lambda)=J(x,u) + \lambda c(x,u) = 4x^2+5u^2+7xu+5x+6u-8 + \lambda (x-u^2-8)$$
	Также найдем градиент и приравняем каждый элемент к нулю:
	$$\begin{matrix}
		\begin{cases}
			\frac{\partial L}{\partial \lambda}=0 \\
			\frac{\partial L}{\partial x}=0 \\
			\frac{\partial L}{\partial u}=0 \\
		\end{cases} &
		\begin{cases}
			x-u^2-8 = 0 \\
			8x+7u+5+\lambda = 0 \\
			10u+7x+6-2\lambda u = 0
		\end{cases}
	\end{matrix}$$
	
	
	\newpage
	
	\section*{Вывод}
	

	
\end{document}